\documentclass{article}

\usepackage[english]{babel}

% Set page size and margins
% Replace `letterpaper' with`a4paper' for UK/EU standard size
\usepackage[letterpaper,top=2cm,bottom=2cm,left=3cm,right=3cm,marginparwidth=1.75cm]{geometry}

% Useful packages
\usepackage{amsmath}
\usepackage{graphicx}
\usepackage{amsthm}
\usepackage{tikz}
\usepackage{tikz-cd}
%\usepackage{biblatex}
\usepackage[colorlinks=true, allcolors=blue]{hyperref}

\theoremstyle{definition}
\newtheorem{definition}{Definition}[section]


\title{Version Space Algebras and Category Theory}
\author{Bailey Wickham}

\begin{document}
\maketitle

\begin{abstract}
We give a category theoretic definition of version space algebras and their operations. 
\end{abstract}

\section{Introduction}
Version Space Algebras are useful as a framework for machine learning in Computer Science, however their associated definitions are limited by this applied context. Many of the constructions can be generalized to be category theoretic. 
                    
Version space algebras provide a set of hypotheses and a set of examples. The example set is used to produce more relevant hypotheses based on updated information. In an applied context these are ranked by an ordering and then used to generate predictions. Due to their development in applied machine learning, the definitions are based entirely in set theory, however these constructs can be translated into a category. 

\section{Version Spaces}
In \cite{short} Lau et al. define a Version Space $VS_{H,D}$ as such: 

\begin{definition}[Version Space] 
A function $f$ is said to be consistent with an example $(a,b)$ if $f(a) = b$. 

Let the hypothesis space $H = \{f | f: A \rightarrow B \}$, where $A$ and $B$ are sets. Let the examples $D = \{(i, o) | i \in A \text{ and } o \in B\}$ Then our version space $VS_{H,D}= \{f | f(i) = o \text{ for all } (i,o) \in D\}$
\end{definition}

The hypothesis space provides an ``ambient space" to work in, or a space of functions which could possibly be models. The examples serve to constrain which functions we are studying, giving a way to refine the ambient space into a specific model.  The examples, $D$, are pairs of elements in the domain and range of these functions for which our model should respect. A version space is the set of the functions in the ambient space that are consistent with the examples. 

In practice the ambient space may be something like "words in our document", or "integers representing row and column numbers", however then only need be functions between (small) sets. Version spaces are used to define all possible actions of a specific type, then constrained to the examples observed in the current document. This is used to generate predictions, where the examples we have used in our document are used to predict future actions.

[] introduces the idea of Version Space Algebras (VSAs) which are version spaces generated by operations on other version spaces. A version space which is defined explicitly, or not defined in terms of operations on other version spaces, is called an \textit{atomic version space}. From a set theory perspective there is no difference between a version space algebra and a version space, there is only a difference in how they are constructed. 

There are four main operations defined on a version space. 
\begin{definition}[Version space union]
    Let $H_1 = \{f | f: A \rightarrow B\}$ and $H_1 = \{f | f: A \rightarrow B\}$ be hypothesis spaces with the same domain and range. Let $D$ be a set of examples. Then $VS_{H_1, D} \bigcup VS_{H_2, D} = VS_{H_1\bigcup H_2, D}$. 
\end{definition}

\begin{definition}[Version space intersection]
Let $H_1$ and $H_2$ be two hypothesis spaces such that the domain of functions in $H_1$ equals those of $H_2$. Let $D$ be a sequence of training examples. The version space intersection $VS_{H_1, D} \bigcap VS_{H_2, D} = VS_{H_1\bigcap H_2, D}$.
\end{definition}

\begin{definition}[Version space intersection]
Let $H_1$ and $H_2$ be two hypothesis spaces such that the domain of functions in $H_1$ equals those of $H_2$. Let $D$ be a sequence of training examples. The version space intersection $VS_{H_1, D} \bigcap VS_{H_2, D} = VS_{H_1\bigcap H_2, D}$.
\end{definition}

\begin{definition}[Version space transform]
Let $\tau_i$ be a function mapping elements from the domain of $VS_1$ to the domain of $VS_2$, and $\tau_o$ be a one-to-one mapping of elements in the range of $VS_1$ to elements in the range of $VS_2$. Version space $VS_1$ is a transform of $VS_2$ iff $VS_1 = \{g : \exists f \in VS_2, \forall i,  g(i) = \tau_o^{-1}(f(\tau_i(i)))\}$

While not provided in \cite{short}, a transform can be represented as a commutative diagram. For all $f$ this diagram commutes, where $\tau_o$ is injective and $A_i$ is the domain of $H_i$ and $B_i$ is the range of $H_i$. 

% https://q.uiver.app/?q=WzAsNCxbMCwwLCJBXzEiXSxbMCwxLCJBXzIiXSxbMSwwLCJCXzIiXSxbMSwxLCJCXzIiXSxbMCwxLCJcXHRhdV9pIl0sWzAsMiwiZyIsMl0sWzMsMiwiXFx0YXVfbyJdLFsxLDMsImYiXV0=
\[\begin{tikzcd}
	{A_1} & {B_2} \\
	{A_2} & {B_2}
	\arrow["{\tau_i}", from=1-1, to=2-1]
	\arrow["g"', from=1-1, to=1-2]
	\arrow["{\tau_o}", from=2-2, to=1-2]
	\arrow["f", from=2-1, to=2-2]
\end{tikzcd}\]

\end{definition}

\section{Category Theory}
In the previous section we defined a version space and a set of operations on it. In this section we will redefine version spaces in categorical terms. First we define our objects which are version spaces. 

\begin{definition}[Version Space (Category)]
Let $A$ and $B$ be small sets. Then a hypothesis space $H \hookrightarrow Hom_{Set}(A, B)$. Let the examples, $D$, be a set of pairs $\{(a,b) : a \in A, b \in B\}$ $(a,b)$ that project via maps $\partial_0, \partial_1$ into $A$ and $B$ respectively.

Let $\Gamma_f$ be the graph functor: $\Gamma_f = \{(a,b) | b = f(a)\} \subseteq A\times B$. We can define a version space as the set  $VS_{H,D} = \{f : D \rightarrow \Gamma_f \text{ is injective}\}$. We say $D$ is consistent with $f$ if $D \rightarrow \Gamma_f$ is injective. A version space is all of the functions which are consistent with all of the examples: $VS_{H, D} = \{f : \text{ D is consistent with } \Gamma_f $ \}

Define the projection $\pi_D : VS_{H,D} \rightarrow D$, and $\Gamma_H = \{\Gamma_f : f \in H\}$
We can phrase this as a diagram: 
% https://q.uiver.app/?q=WzAsMyxbMCwwLCJWU197SCwgRH0iXSxbMiwyLCJcXEdhbW1hX0giXSxbMiwwLCJEIl0sWzAsMSwiXFxHYW1tYSJdLFswLDIsIlxccGlfe0R9Il0sWzIsMSwiXFxpb3RhIiwyLHsic3R5bGUiOnsiYm9keSI6eyJuYW1lIjoiZGFzaGVkIn19fV1d
\[\begin{tikzcd}
	{H\times D} && D \\
	\\
	&& {\Gamma_H}
	\arrow["\Gamma", from=1-1, to=3-3]
	\arrow["{\pi_{D}}", from=1-1, to=1-3]
	\arrow["\iota"', dashed, from=1-3, to=3-3]
\end{tikzcd}\]

Note that since $\Gamma_H$ is a set of sets, $\iota$ must be defined as such: $\iota: D\rightarrow \{D\}$. Therefore a $VS_{H,D}$ is the set of $D$ such that $\iota$ is injective and the diagram commutes.

\end{definition}

This construction mirrors the set theoretic construction given by [] et al. Union and Intersection are defined almost identically to the set theory version. 


\begin{definition}[Category of VSAs]
Let \textbf{VSA} be the category of version spaces. We define the category of all version spaces. 

Objects in our category are of the form $VS_{H,D}$, over any $H$ and any $D$. A morphism assigns $VS_{H_1, D_1} \rightarrow VS_{H_2, D_2}$ if and only if $H_1 \subseteq H_2$.
\end{definition}

\begin{proof}
Associativity of morphisms follows from associativity of the inclusion map. $H \subseteq H$, so there exists an identity morphism for every object. Therefore \textbf{VSA} forms a category.
\end{proof}

\begin{definition}
The union and intersection definitions are the same as before: 
\begin{gather*}
    VS_{H_1, D} \bigcup VS_{H_2, D} = VS_{H_1\bigcup H_2, D}  \\ \text{ and } \\
     VS_{H_1, D} \bigcap VS_{H_2, D} = VS_{H_1\bigcap H_2, D}  
\end{gather*}
\end{definition}

For transformations, 



\begin{thebibliography}{9}
\bibitem{short}
Tessa Lao, Pedro Domingos, Daniel S. Weld, Version Space Algebra and its Application to Programming by Demonstration
\end{thebibliography}


\end{document}